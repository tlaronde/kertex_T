\input article
\input manmac
%\magnification=1200
\parskip=1ex

% It starts here...
%
\hrule height 0.2ex
\vskip 1ex
\maintitle{PROTE}
\vskip 3ex
\centerline{Thierry Laronde---tlaronde\AT polynum.com}
\centerline{1.1.0}
\centerline{2023-08-01}
\vskip 1ex
\hrule height 0.2ex
\vskip 1ex

{\raggedcenter\par{Acknowledgements: Philip Taylor and Nelson H. F.
Beebe have made several suggestions and reported several corrections
to the first draft of the manual. Philip Taylor has once more reviewed
this document afterwards. Thank to them, if this draft is
not good, at least it is better! Martin Ruckert has reported some
typos and infelicities in the \PASCAL\ code that got in the way when
trying to directly use the \PASCAL\ program produced. And Phelype
Oleinik has been a great help in narrowing down test samples from the
La\TeX\ format in order to correct the implementation of the new
primitives. Thanks to them all!}\par}

\vfil
\input mytabmat
\vfil\eject
\begindiv The name of the game

@ {\sl Prote} is a French word with the following meaning (translated
from the first sentence of the definition in M.~Bescherelle's
 {\sl Dictionnaire national}, $11^{th}$ edition, 1865):
\iquote{Typography. Title given, in a printing house, to the person who,
under the guidance of the master, leads, conducts and oversees the
typographical implementation of a book.~}

@ It can be seen too as a short name for a word play around 
{\sl proto-typing}, i.e., creating an archetype from some fundamental
elements (primitives: {\it proto}) leading to changing/various
forms---because of protean and Proteus\dots.

\enddiv % The name of the game

\begindiv Implementation principles

@ \Prote\ inherits---obviously and primarily---from D.E.~Knuth's \TeX,
but also inherits from the \NTS\ team' \eTeX, not only the extensions
themselves, but also the principles on which these are based: extensions
are added to \TeX, but switches are required to activate these
extensions; in the absence of these switches, the behavior of the
engine is compatible with \TeX.

However, while in \eTeX\ one can selectively activate or not each 
feature, in \Prote, once activated, all \Prote\ additions are there.

@ \Prote\ has three modes of operation:

(1)~In \TeX\ compatibility mode, it fully deserves the name \TeX\ and
there are neither extended features nor additional primitive commands.
That means in particular that \Prote\ passes the {\tt TRIP} test
without any restriction.  There are, however, a few
minor modifications that would be legitimate in any implementation of
\TeX\note{For the moment, these come from \eTeX.}.

(2)~An initial asterisk (in {\tt INITEX} mode) switches to \eTeX\
extended mode, giving access to \eTeX\ additional primitive commands
and extended features.

(3)~Two asterisks (in {\tt INITEX} mode) switch to \eTeX\ and \Prote\
features, i.e., the \Prote\ extensions being unconditionally added to
\eTeX.

@ Following the example of the \NTS\ team, we have limited the
modifications of the \TeX\ engine proper to the minimum, and added the
bulk of the supplementary features to the 53rd section, the 54th
still being a hook for host system changes.

The \Prote\ change file {\tt prote.ch} thus extends \TeX\ and \eTeX\ but
does not depend on the ker\TeX\ distribution, even if it was developed
in its midst. Hence this change file can be used by anyone, having to be
applied after the \eTeX\ change file and before any change file needed
for the host system adaptations.

Some extensions may require a call to the host system for things that
\PASCAL\ does not provide. A basic implementation of these routines,
based on the |fix_date_and_time| example, is provided, yielding a 
syntactically correct ``answer'', but one that may be semantically
useless or semantically correct but obtained in an inefficient way
(for example, finding the size in bytes of a file). These are marked
as {\sl system dependencies} and are to be adapted as usual.

@ If \eTeX\ introduces modifications also at the semantic level,
for the moment the \Prote\ primitives are only at the syntactic
level (in the early process when mastering the commands) or, late,
supplementary information in the DVI produced.

In order to clarify things, we speak of {\sl command classes} for
the main command level ({\it cf}. m.207 to m.210 in {\sl \TeX: The Program}). 
Every variant in these classes has to be identified in the class by a
command modifier.

We do not add any new command classes, so our primitives are only variants
(instances) of the \TeX\ existing command classes.

When adding a new primitive, the first decision to make is in which
command class to add it, because it drives the behavior: expandable
or not; value to be expanded by |\the| or not; read-only or read---write
(this is the distinction between parameters and external variables).

The following principles have been followed:

(1)~When a primitive corresponds to a read-write variable, the variable
has been added as a parameter in the corresponding |eqtb| region,
integer (if the units, even if defined, is not specified at expansion)
or dimension. These are |assign_*| variants.

(2)~When the primitive corresponds to a read-only variable but whose
value can be expanded by |\the|, it is a |last_item| variant.

(3)~When the primitive expands to something that is externally given
(and is not expanded by |\the|), it is a |convert| variant.

(4)~When the primitive is a conditional, it is obviously a |if_else|
variant.

(5)~When the primitive changes (or may change) the definition of the
following token, it is a |expand_after| variant.

(6)~When the primitive is really orthogonal, it is an |extension|
variant.

@ In \TeX, there is a lot of ``numbering'' in the sense of encoding
various relations between objects in the relations between the integers
used to designate them. For example, there are command codes and
command code modifiers, and, even if we do not introduce new commands,
we must not step on \TeX\ or \eTeX\ toes with the command
modifiers.

@ The numeric routines dealing with pseudo-random numbers are
John Hobby's implementation in \MP\ of D.E.~Knuth's algorithms from
Section 3.6 of {\sl The Art of Computer Programming}, marginally adapted
to fit in \TeX.

\enddiv % Implementation principles

\begindiv Style

@ We have tried to stick to D.E.~Knuth style and to the \NTS\ team
style. Additions (not deviations) are indicated below.

@ \PASCAL\ does not permit block scope variables inside the routines. It
is an unfelicity because the variables declared in the
routines have possibly two different uses: some really general
variables that drive the general process (these are the
variables mainly defined in the original \TeX\ routines) or variables
that are indeed needed by the cases and have a limited block use.

In order to differentiate, following D.E.~Knuth practice, we add in the
comment aside the added variables {\tt general purpose} in order to
indicate that the values of these variables can be set in the cases
blocks without interfering with the general process of the routine.

We try also to give such added variables a sufficiently general meaning
(a single letter), so that they can be used for different purposes
according to their type.

\enddiv % Style

\begindiv Testing

@ D.E.~Knuth has designed a torture test that is a \TeX\ command file in
order to check the conformity of the TeX engine. The torture test is
named {\tt TRIP}. What is tested is a version of the {\tt INITEX}
instance of the \TeX engine, with some defined parameters for
compilation.

So there are two things to be distinguished: the name of the command
file pushing the engine in all dark corners (for \TeX: {\tt TRIP}) and
the version of the engine tested (in ker\TeX, the corresponding
{\tt INITEX} for the \TeX version is called {\tt triptex}).

@ \eTeX\ adds more than a simple torture test, called {\tt ETRIP}: it
adds the constraints that the {\tt INITEX} instance of the \eTeX\ engine
(called {\tt etriptex} in ker\TeX) shall be \TeX\ compatible, this means
shall pass the {\tt TRIP} test, in both compatible mode and \eTeX\ mode.

@ \Prote, in the same spirit, imposes that the {\tt INITEX} instance of
\Prote\ shall pass the {\tt TRIP} test in the three modes: \TeX\
compatible, \eTeX\ compatible and \Prote; and the {\tt ETRIP} test in
two modes: \eTeX\ compatible and \Prote.

To this is added, not a torture test at the moment, but tests for all
the primitives added in order to verify that they conform to the API
contract defined below.

\extra{In ker\TeX, one will find the Bourne shell scripts {\tt ck\_tex},
{\tt ck\_etex} and {\tt ck\_prote}, in the directory {\tt 
\$KERTEX\_BINDIR/adm}, to check the conformance. One can refer to these
files for further information.}

\enddiv % Mise sur la sellette

\begindiv System integration

@ There are some added system dependencies---things that are not
provided by standard \PASCAL\ and, hence, have to be provided by system
dependent code.

These are file or time/date related, or bitwise processing related
(MD5 hash computation).

\begindiv Exchanging with the system

@ In order to try to protect the inner code of \TeX, when we need to
exchange with some externally provided routine, we use a separated
dedicated array of
bytes called |xchg_buffer|\note{For system integrators: please remember
that the underscore is not allowed by standard \PASCAL\ and that {\tt
tangle} suppresses the underscores in the identifiers finally
produced.}. Its size is specified by a constant (m.11)
|xchg_buffer_size|, whose value is checked to provide at least what is
expected by the default code.

If the value is incorrect, the |bad| code $51$ is returned and {\tt
iniprote} terminates if requested to switch to \Prote\ mode---the check
is not made in other modes since the array is only used for \Prote\
supplementary features.

The array is an array of bytes (|eight_bits|), in order to be used for
whatever interpretation of bytes. The values, if characters, have to be
in the host encoding, the internal code making the transcription to
\TeX\ ASCII when needed.

A global variable |xchg_buffer_length| shall be defined with the
last index in the buffer with valid data, whether set by the internal
code when passing data to external routines, or set by the external
routine to pass data to the internal code. Remember that the index
starts at $1$ (and not $0$ like in normal C code). If |xchg_buffer_length|
is $0$ this does mean that there is no data. Even if the array is
representing a string, it is neither NUL or space terminated: the
|xchg_buffer_length| gives the size.

\enddiv % Exchanging with the system

\begindiv Date and time reference

@ The |fix_date_and_time| routine in \TeX\ shall fix the reference
moment for the timer---see primitives |\elapsedtime| and |\resettimer|---
as well as the creation date---see primitive |\creationdate|.

Since \PASCAL\ can not provide anything useful here, the reference date
and time are D.E.~Knuth' reference date: 4 July 1776 at noon (local
time).

@ For the creation date, if the environment variable |FORCE_SOURCE_DATE|
is $1$ and |SOURCE_DATE_EPOCH| is set, the creation date and the values
for |\year|, |\month|, |\day| and |\time| are set from the definition of
|SOURCE_DATE_EPOCH| and the time zone is UTC. This is of some use only
if the system integrators have provided a better version of the default
code.

@ The timer is not a clock: \Prote\ does not interrupt regularily to
increment an internal counter: the elapsed time is returned at request,
by taking the difference, in scaled seconds, between the moment of the
call and the reference moment. This value can not exceed \TeX\ scaled
integer that is |infinity|.

Hence, without system support, resetting the timer (redefining the
moment of reference) will not change what the default implementation
returns, i.e., |infinity|.

The creation date is not stored in \Prote; it is a system fixed value
corresponding to the reference date, retrieved at request.

\enddiv % Date and time reference

\enddiv % System integration

\begindiv Identifying \Prote\ mode

@ The specific |\Proteversion| (an integer that can be used in
assignment or dereferenced by |\the|) and |\Proterevision| (expanding to
a string) primitives can be tested to detect the \Prote\ mode. They
have the very same nature as the corresponding |\eTeXversion| and
|\eTeXrevision|.

\enddiv % Identifying \Prote\ mode

\begindiv \Prote\ additional parameters and primitives

@ The aim of the present section is to give a definition of the {\sl
usage} of the new primitives, i.e., the syntax of the call and the
result. In what follows, the syntax definitions shall agree with the
conventions of the \TeXbook---see {\sl chapter 24} and {\sl chapter 25}.

For example, |<general text>| means a text starting with optional
spaces and/or |\relax|, a (possibly inserted) |<left brace>|,
balanced text and a |<right brace>|; and it is subject to expansion
or not depending on the command class.

@ We will not repeat below the following features: keywords are case
insensitive (this is a \TeX\ feature); an |<integer>| can be given by
whatever mean is legitimate at this moment.

@ For all the file related primitives, if the file can not, for whatever
reason, be opened, there is no error set, no dialog with the user, but
the primitive expands to nothing (it is a no-op or ``returns'' an empty
string).

{\it De facto}, La\TeX\ uses for example |\filesize| in order to test
for the existence---or at least the readability, since search routines
intervene---of the file: if the result is the empty string, it is not
available.

\begindiv \eTeX\ missing

@ \eTeX\ uses |expand_depth| and |expand_depth_count| but these are not
defined anywhere.

\Prote\ adds them, making |expand_depth| an integer parameter and hence
a primitive.

@ |\expanddepth|. ({\sl integer parameter})\enskip
The maximum of expansion nesting. It defaults to $10000$ (simply to take
the current value in \TeX{}live) and can be set to any value by the
normal integer parameter assignment.

\enddiv % \eTeX\ missing

\begindiv States

@ |\shellescape|. ({\sl last\_item class})\enskip
It expands to $1$ if (PDF\TeX) |\write18| is enabled, $2$ if it is 
(PDF\TeX) restricted, and 0 otherwise. In \Prote, it hence always
expands to $0$ because there is no such feature.

\enddiv % States

\vfil
\break
\begindiv Conditionals

@ |\ifincsname|.\enskip
The conditional is true if evaluated inside
|\csname|\dots|\endcsname|, and false otherwise.

@ |\ifprimitive <control sequence>|.\enskip
The conditional is true if the following control sequence is a
primitive and has its primitive meaning, and false otherwise.

\enddiv % Conditionals

\begindiv Changing definition or expansion

@ |\primitive <control sequence>|. ({\sl expand\_after class})\enskip
If the control sequence name corresponds to the name of a primitive,
switches to the primitive meaning. If not, this is a no-op (the control
sequence is executed as is).

\extra{At the moment, the primitive indeed switches the meaning if the
following token is a control sequence whose name corresponds to a
primitive, but does not raise error if the following token is even not a
control sequence. Should it? Furthermore, should the primitive be
locked, made unredefinable, in order to always have the ability to go
back to the primitive meaning?}

@ |\expanded <general text>|. ({\sl expand\_after class})\enskip
The |<balanced text>| is expanded except for (\eTeX\ feature) protected
control sequences and the resulting token list is sent back for 
scanning.

\enddiv % Changing definition or expansion

\begindiv Strings

@ |\strcmp <general text> <general text>|. ({\sl convert class})\enskip
The two |<balanced text>| arguments (expanded) are converted to \TeX\
strings, i.e., |ASCII_code| strings. The primitive expands to a
number indicating the relative |ASCII_code| encoding order of the
first string relative to the second: $-1$ if the first one is ranked
before; $0$ if the two strings are equal; $1$ if the first string is
ranked after. Empty strings are allowed for either argument or both,
and two empty strings are equal.

\crux{The strings are \TeX\ internal strings, that is {\tt ASCII\_code}
strings, i.e., the system encoding is not taken into account, the result
is sorted according to the |xord| conversion. And the |<balanced text>|,
being expanded (for spaces, it would be true even if not expanded),
is not a verbatim copy of the input.}

\enddiv % Strings

\begindiv Random numbers

@ |\randomseed|. ({\sl last\_item class})\enskip
|randomseed| is an integer. Its value can be used as a |<number>| or its
expansion obtained via |\the|. It is the value used for seeding an array
of numbers for pseudo-random numbers generation.

\crux{This is not an integer parameter precisely because
assigning a new value has to trigger the redefinition of the array for
pseudo-random numbers generation.}

@ |\setrandomseed <integer>|. ({\sl convert class})\enskip
Set the value of |\randomseed| to this specific value. This has the
effect of regenerating an internal array of numbers that will
serve as the source of pseudo-random numbers for the ``deviate''
following primitives.

@ |\normaldeviate|. ({\sl convert class})\enskip
Expands to one pseudo-random normally-distributed integer according
to a mean of $0$ and a standard deviation of $65536$. That is,
numerous successive calls of the (underlying) function (with the
same unchanged seed) will yield, 68\% of the time, an integer in
the range $[-65536,65536]$ (one standard deviation away from the
mean) and about 95\% of the time an integer within two standard
deviations, and 99.7\% within three.

@ |uniformdeviate <integer>|. ({\sl convert class})\enskip
Expands to one pseudo-random integer between $0$ (inclusive) and the
|<integer>| (exclusive)---the integer may be negative---, with a
uniformly-distributed probability.

\enddiv % Random numbers

\begindiv Date and time

@ |\creationdate|. ({\sl convert class})\enskip
Expands to the date of creation corresponding at the moment when
the routine |fix_date_and_time| was called.
The format is |D:YYYYMMDDHHmmSSOHH'mm'|,
`O' being the relationship of local time to UTC, that is `-' (minus),
`+' or `Z'; HH followed by a single quote being the absolute value
of the offset from UTC in hours (00--23), mm followed by a single
quote being the absolute value of the offset from UTC in minutes
(00--59). All fields after the year are optional and default to
zero values. If the environment variable |FORCE_SOURCE_DATE| is $1$
and |SOURCE_DATE_EPOCH| is set, the creation date has been set from
the |SOURCE_DATE_EPOCH| definition and the date is in UTC.

\extra{Because there is no support for date related routines in standard
\PASCAL, the default implementation always sets the date and time
relative to 4 July 1776, noon, local time.}

@ |\elapsedtime|. ({\sl convert class})\enskip
Expands to the scaled integer value of the elapsed time since the
reference moment, the (not expressed) unit being the scaled second, i.e.,
$1/65536$ second. It is a \TeX\ scaled integer in the defined range,
that is the maximal value is $2^{31}-1$; the elapsed time can not
reach $32767$ seconds and in this case, the maximal value of $2^{31}-1$
is returned and shall be considered as invalid time.

\crux{Because there is no support in standard \PASCAL\ for these time
related routines, infinity is always returned in the default
implementation, even if the reference moment is reset by
resettimer---because there is no internal clock
incrementing and the time is always computed and returned only on
demand.}

@ |\resettimer|. ({\sl extension class})\enskip
Reset the elapsed time reference moment with the moment of the
processing of the primitive,

\crux{In the default implementation, because standard \PASCAL\ doesn't
provide the necessary routine, the primitive is a no-op. There is no
internal counter, so elapsedtime continues returning infinity.}

\enddiv % Date and time

\begindiv DVI related

@ |\pageheight|. ({\sl dimen parameter})\enskip
Specifies the height of the current DVI paper (the primitive is
misnamed) on which the \TeX\ page will be printed. It is only
DVI related and is used to compute the coordinates saved by |\savepos|
in the integers |\lastxpos| and |\lastypos|. Defaults to $0$ and
no verification is made and no error is raised if the value is
nonsense.
 
@ |\pagewidth|. ({\sl dimen parameter})\enskip
{\it De dicto}.

@ |\savepos|. ({\sl extension class})\enskip
This primitive, when a vlist or hlist is shipped out and DVI produced,
saves the x and y current position in the DVI page (paper) when called,
this position being given, in scaled points, in a coordinate system
with the origin set to the lower left corner of the paper, whose
size should have been defined with the (misnamed) |\pagewidth| and
|\pageheight|. In DVI, the upper left corner of the page is $1in$
to the left and $1in$ up relative to the \TeX\ origin.

\extra{\Prote\ adds a {\tt special} in the DVI file consisting in
{\tt \_\_PROTE\_SAVEPOS=index XPOS=number YPOS=number}, index being the
ordinal of the call, incremented each time the primitive is called,
first being $1$. This permits the user to not be limited to one {\it
hic} hint by using the utility {\tt dvitype} to retrieve all the saved
positions. Note that the prefix {\tt \_\_PROTE\_} (case insensitive) shall
be considered reserved.}

@ |\lastxpos|. ({\sl last\_item class})\enskip
Expands to the integer representation of the last DVI x saved value. The
coordinates system is described above, and the value is relative to the
non expressed unity: scaled point.

@ |\lastypos|. ({\sl last\_item class})\enskip
As the previous but for the y.

\enddiv % DVI related

\begindiv File-related primitives

@ For these primitives, a |<balanced text>| (expanded) is converted
to a string that is used as the name of the file to open. The
handling is the same as for the |\input| primitive
that is, it is subject to path searching (done by the implementation)
and to the addition of the |.tex| extension is there is
none (one can not input in \TeX\ a file without an extension). But,
contrary to the |\input| primitive, no error is raised if the file can
not be opened: the primitive is then a no-op (i.e. expands to the
empty string).
 
@ |\filemoddate <general text>|. ({\sl convert class})\enskip
Expands to the modification date of file |<balanced text>| (expanded)
in the same format as of |\creationdate| (see above).
If the |SOURCE_DATE_EPOCH| and |FORCE_SOURCE_DATE| environment
variables are both set, |\filemoddate| returns a value in UTC,
ending in Z. The modification time is about the contents, not the
metadata (|mtime| in Unix, not |ctime|).

\crux{Because this information is not available in standard \PASCAL,
a routine has to be provided by the system if it supports this. The
default implementation returns nothing.}

@ |\filesize <general text>|. ({\sl convert class})\enskip
Expands to the size in bytes of file |<balanced text>| (expanded).

\extra{If the file can not be opened, no error is raised and the
primitive expands to nothing (empty string).}

@ |\filedump [offset <integer>] [length <integer>] <general text>|.
({\sl convert class})\enskip
Expands to the dump of the first length bytes of the file |<balanced
text>| (expanded), in uppercase hexadecimal format, starting at the
offset specified or at the beginning of the file if offset is not given.
If length is not given, the default is zero, so |\filedump| expands to
nothing---the same is true for any file related error; but giving an
improper offset or length value raises an error.

@ |\mdfivesum [file] <general text>|. ({\sl convert class}).\enskip
If the keyword |file| is given, expands to the
MD5 of file |<balanced text>|; without |file|, expands to
the MD5 of the |<balanced text>|. In both cases, |<balanced text>| is
expanded. The MD5 is given as a 32 uppercase hexadecimal characters
string. An empty text or an empty file are valid and lead to the
MD5 hash of no data. This same value is returned, in |scroll_mode|,
if there is an error with the file.

\extra{Because the computing of MD5 involves bitwise operations, that
\PASCAL\ doesn't---at least: easily---provide, the default code expands
to nothing---the same happens if there is a problem with the file
opening.}

\enddiv % File-related primitives

\enddiv % Added primitives

\bye
