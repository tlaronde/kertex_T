\input article
\input manmac
%\magnification=1200
\parskip=1ex

% It starts here...
%
\hrule height 0.2ex
\vskip 1ex
\maintitle{The bytes stream as input: data and filenames}
\vskip 3ex
\centerline{Thierry Laronde---tlaronde\AT polynum.com}
\centerline{2022-04-17}
\vskip 1ex
\hrule height 0.2ex
\vskip 1ex

\resume{Thanks to Phelype Oleinik for providing a \TeX\ test file
for the implementation and for commenting on the early draft!}
 
\begindiv The input in \TeX\ related programs

@ The Donald E. Knuth's digital typography programs accept as input 
lines as streams of unsigned integers in the range $[0,255]$. Whether
these, in the programs, are octets (bytes) or not is an implementation
detail. What the programs see is a sequence of unsigned integers in
this range.

Since the \TeX\ and \MF\ programs are compilers/interpreters chewing
``text'', they have to understand at least a subset of this integer
range as mapping to defined "letters". The convention adopted for
the encoding of this subset is the ASCII one, at least for the
printable characters in the subrange $[32, 126]$, that is from
space to tilde (see module $21$ of {\tt tex.web} for example).
There are $3$ other special values: the {\tt null\_code} ($0$); the
{\tt carriage\_return} ($13$ i.e. ASCII carriage return, not new
line) and the {\tt invalid\_code} ($127$) (see module $22$ of {\tt
tex.web} for example). But if these three values are used internally,
they are not expected to appear by default in the input.

\extra{This use of a stream of bytes and of a (partial) ASCII convention
leads naturally to the conclusion that UTF-8 also could be handled
by \TeX\ (the program) without very involved modifications---but
the devil is in the details: see below. This adaptation to accept also
the UTF-8 way of encoding wider integers as a stream of bytes (but not
in any mapping assumption, except for ASCII, even not Unicode) will be
done some day in ker\TeX, but is not done yet.}

@ Since the programs expect at least this ASCII convention for the
subrange specified, two arrays are used to allow a conversion between
the system encoding and this \TeX\ encoding. 

The programs get input files line by line, and the routine inputing
a new line in the buffer scanned by the programs, translates the
system characters to the \TeX\ convention using the so-called {\tt xord}
array.

When the programs send something to the system (including on the
terminal), the other array, so-called {\tt xchr}, is used to translate
from the \TeX\ convention to the system encoding.

\crux{We have {\sl not} written: the {\sl converse} or {\sl reverse}
array in order to not give the false impression that the mapping
conversion between ``chars'' is, implicitely, reversible i.e. that
the mapping from $[0,255]$ to $[0,255]$ is injective and thus,
the sets being finite and of equal cardinal, the translation 
function is bijective: in general, it shall not be taken for granted
that $xchr[xord[i]]==i$. For example, in the written source, this
does not hold. Since these arrays have been created to allow
adaptation according to local (system) idiosynchrasies, it depends
on the implementation.}

\extra{In ker\TeX, the arrays are the identity arrays: the mapping
yields the very same octet in both ways. This allows to input
whatever, but the user level (macro package using the engines) has
to provide handling for the supplementary values outside the ASCII
printable range that may appear as input.}

@ It shall be noted that since this ASCII partial convention is
used as the basis for \TeX\ (here: not the system, but the very
program itself), the result emitted by \TeX\ is in this \TeX\ encoding
for the fonts\note{There are means to create a TFM file for use by
\TeX, refering to a font (for the glyphes) not organized according
to this convention: virtual fonts. In this case, the remapping is done
by the DVI interpreter when a hardcopy is to be produced.}.

@ There is one word upon which an emphasis has to be made: {\sl line}.
The input (since the inputs are text programs) is read line by line, but
whatever is the marker of the end of line is not part of what is scanned
by the program: the {\tt eoln}---being one or several characters---is
removed, as well as trailing ``blanks''.

\crux{In ker\TeX, the trailing ``blanks'' removed are not limited to
``the'' (ASCII) space character, but are whatever bytes are considered
to be space by the C routine {\tt isspace(3)}, that is ``the'' space and
all the binary spaces. If one wants some trailing spaces to be
kept (because one uses them as active characters for example), they must
be followed by a non space character (for example, a comment character)
before the end of line in order to be retained.}

\extra{In ker\TeX, the C library {\tt fgets(3)} routine is used
in order to let the system convert whatever is an end of line marker
to $\bslash n$. The question arises about the portability of a
\TeX\ or a \MF\ script, written on a system and with an editor that
will use say the carriage return as the end of line marker (not
followed by the ASCII end of line; this is the case for old
Macintosh), if such a script would be fed to a \TeX\ or \MF\ instance
on a system with a differing convention: the lines will not be
detected the same way, and in this case the whole file would be absorbed
and an error will probably be raised because the size of the buffer
is not sufficient to hold the line\dots It shall be noted that some
network utilities, like {\tt ftp(1)}, have a special mode so that
the transfer of files declared as text would change the end of line
markers. But the fact is that, contrary to common belief, a ``text''
file is not absolutely portable.}

\begindiv Filenames

@ In some sense, there are four different ways a filename is handled
in the \TeX\ or \MF\ (or derived) program:

(1)~The {\tt poolname} is a file whose name is given ``as is'',
without any translation (with {\tt xord} or {\tt xch}) or scanning,
to the opening routine since it is needed before everything---think
of bootstrapping a kernel.

(2)~A file format, that is a pre-compiled version of a set of macros and
predefined memory areas, is given with a name taken at the very
beginning of the job. This name (even the one defined as the default
name) is translated using the {\tt xord} array but is handled by
a special routine.
 
(3)~Other input files have their name taken from the stream of
text instructions, meaning that, depending on the implementation,
there is a first translation of the system bytes to the \TeX\ convention
to feed the buffer; followed by an expansion of what is given and
a tokenization of the result with ``the'' space being used as the
delimiter and the filename being defined as the first such token;
and then a conversion of this defined name (with, perhaps, an extension
added) back to the system encoding.

(4)~And finally, when an input file is not found, if interactive,
there is a prompt to give the file name; but this is not subjected
to expansion, contrary to the third case above, but still with a
space tokenization.

In the special first case of the poolname, spaces could be in the
name, while they can't exist in the other cases due to the space
tokenization.

\extra{Blanks in filenames---or paths---are definitively not something
the author likes or recommends and a \TeX\ implementation could
accept---it was the case in ker\TeX\ before this very
modification---spaces in paths searched but not in the portion of the
filenames passed to the programs. But we enforce no policy: we provide
rope\dots}

@ A filename given to the |\input| primitive has always an extension. If
the extension is not given, the default one is added (this is ``.tex''
for the \TeX\ engines; ``.mf'' for a \MF\ script; ``.mp'' for a \MP\
script).

So one can input a file with a not default extension but can not input
a file without an extension.

@ Engines derived from \TeX\ (here: the program itself) have
allowed to specify a filename including spaces by specifying a string
enclosed in double quotes; and a further extension has been developped
to allow to pass an argument in the more common way using grouping.

This is allowed not as a \TeX\ extension but as an implementation
detail. This is why it is not provided in \Prote\ but as a separate
change file to be applied when making implementation modifications.

\crux{This extension impacts only the input primitive, and not the
{\tt fmt} case: in ker\TeX, a fmt (or whatever a digest is) name has
no spaces.}

@ But there is still a problem to consider:

If what \TeX\ (the program) considers is ``text'', meaning that the
integers considered are supposed to map to characters in one or
several languages, a filename has not to be in some language: it
is mainly, in filesystems, a C nul terminated string with some byte
values not accepted (the nul character is never accepted; the
directory separator `/' is not accepted either; but other
characters/bytes are forbidden in some filesystems while being valid
in other ones).

To take the example of UTF-8, if inputing a text in this encoding could
make sense, a filename in some filesystem could well be a string of
bytes that is not a valid string for UTF-8.

This does mean that a filename should have a specific treatment. But
this specific treatment would be done in the implementation added 
glue, without modifying the core of the engines' code.

\enddiv % Filenames

\enddiv % The input in \TeX\ related programs

\begindiv System integration

@ The glue code for opening a file has to be adapted in order to be able
to deal with whatever string is given as a filename not relying on
space characters as sentries.

In order to do so, in ker\TeX, |name_of_file| and |name_length|
are defined by the translated Pascal code and the glue code takes the
range $[1, name\_length]$ in the bytes vector |name_of_file|
as the definition of the ``as is'' absolute or relative pathname
to open---|name_of_file| is defined so that there is always room for
a supplementary nul byte, i.e. the C end of string character; but
this nul byte is not added by the Pascal code.

This pathname has already been put in the system encoding by the
calling code.

@ What is called an {\tt area} in the programs, are directories in
ker\TeX. The filenames are searched according to path specifications
taken from the environment.

Files read may be searched. Files written are written without alteration
to the pathname given, meaning are written relative to the (process)
current directory if the dirname is not absolutely qualified
(this is in accordance with what is specified, m. 511, in
\TeX, {\sl The Program}: ``{\sl The file area can be arbitrary on input
files, but files are usually output to the user's current area.}''; in
ker\TeX, the user current area is the (process) current working directory).

If a filename is given as an absolute path, beginning by the root path
separator, the filename is not searched. The same is true for a filename
given with a leading current directory specification (``./''): taken
as is without any search. Unspecified relative filenames are searched
according to a path search specification defined in the environment
variable {\tt KERTEXINPUTS}. It is a semi-colon sequence of paths,
searched in this order, with additional chunks maybe interpolated (ex.:
when the set of macros is the La\TeX\ one, the {\tt virprote} instance
is named {\tt latex} and {\tt latex} as a subdir is added automatically
to the searched paths.

If {\tt KERTEXINPUTS} is not defined in the environment, a default
definition is used, putting the (process) current directory as the first
directory to search, and the system ones to be searched after.

For other kinds of files to search (not |\input|'ed ones), there are
other environment variables following the same principles.

\enddiv % System integration

\begindiv Modification of the input primitive

@ The modification adds other ways to define a filename for the |\input|
primitive, meaning that the traditional way still works.

They are now three ways to define a filename for the |\input| primitive:

(1)~A filename given as a recursive expanded text argument enclosed in
grouping characters (generally: braces), the initial left one being
mandatory (contrary to the common definition of |<general text>|). In
the result, the double quotes are all eliminated (and not replaced by
anything). Contrary to the traditional behavior, such an argument can
call |\input| in its text (one can input the content of a file as the
name of a file to input...);

(2)~A filename given as the expanded text between double quotes, the
double quotes being removed. The last double quote is optional: an
end of line ends the filename;

(3)~The traditional space tokenizing if none of the two previous
ways applies (expansion is done; but the file name is the first token
in the result with space tokenization).

There is one caveat: as for the general text, an initial sequence of
optional blanks and/or |\relax| is ignored (absorbed), and only the
first non such following command is taken into account to switch
between the three ways of defining the filename.

@ |\input [blanks_or_\relax*]
{ <recursive expanded text with mandatory left brace>
OR `"' <expanded text> [`"']
OR <first token of expanded text with space tokenization> }|
({\sl input class})\enskip
Initial spaces or |\relax| are ignored; the first other character is
taken to switch between the three ways of defining the filename and
tested in this order: a |<left brace>| defines the filename as the
expansion fo the general text given as an argument to the primitive,
|\input| commands allowed inside the argument (recursive call) and all
double quotes being removed from the result;
if not and the first character is a double quote, the filename is 
taken as the traditional expansion of the text between the double
quotes, the double quotes being removed (the end of line is considered
as the closing double quote is there was none---one can not input
a newline character in a filename, and neither a double quote); if
neither, the traditional way of defining the file name is taken,
i.e. expansion (|\input| not allowed) and space tokenization.

@ In case of error and prompting, there is still no expansion done, but
it is allowed to provide a name between double quotes, i.e. to specify
a name with blanks.

\enddiv % Modification of the input primitive

\bye
