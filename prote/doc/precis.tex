\begindiv Pr\'ecis

@ If there is a common base for the Pascal language---even a standard---
the compilers differ greatly. But indeed, as numerous (and famous)
others have already written, Pascal has several infelicities and
limitations.

\DEK\ has written \TeX\ so that the Pascal written can be easily
translated to any other language---for example C as is done with web2c
and its derivative in ker\TeX: {\tt pp2rc} (this stands for Pseudo
Pascal to Raw C). But a lot relies too on {\tt TANGLE}, that circumvents
some of the notorious limitations of Pascal.

Hence the section ``How to Read a Web File'' in ``\TeX: The Program''.
But some additionnal notes could be added.

@ {\tt TANGLE} treats what we call strings in a special way. There is a
difference between strings delimited by double-quotes and strings
delimited by single-quotes ({\tt TANGLE}, \P 39):

\beginquote
Strings are treated like identifiers; the first character (a double-quote)
distinguishes a string from an alphabetic name, but for {\tt TANGLE}'s purposes
strings behave like numeric macros. (A `string' here refers to the
strings delimited by double-quotes that {\tt TANGLE} processes. \PASCAL\
string constants delimited by single-quote marks are not given such special
treatment; they simply appear as sequences of characters in the \PASCAL\
texts.)
\endquote

Here, in fact, a string enclosed in double-quotes will be referred to
by a pointer that is only an index. So keep in mind that a string
enclosed in double-quotes will appear in fact as an integer.

When, in the code, the \PASCAL\ |write| and |writeln| procedures are
used, the argument is a string between single quotes.

When the special pool of strings is used, there are special written
routines using in fact the index (justly called a pointer) and they have
the prefix |print|.

@ Furthermore, {\tt TANGLE} starts allocating pointers (indices) at $256$
({\tt TANGLE} \P 42). The $[0..255]$ range is not allocated by {\tt
TANGLE}, but is used in \TeX\ (the program) for the single letter
strings.

The double-quoted strings encountered by {\tt TANGLE} are then given an
index $>= 256$, and are put, one string by line, in the {\sl pool} file,
that has a special format. The maximum length of a string (of bytes) is
$99$; every record in the {\sl pool} file is a line beginning with
a decimal number expressed as 2 digits and immediately followed by the
string. The first record corresponds to the index $256$. The last line
in the {\sl pool} file is a checksum produced by {\tt TANGLE} and
recorded in the program tangled so that loading the {\sl pool} file it
can have a mean to verify that it is (probably if nobody has
deliberately tampered with) a corresponding table.

@ One supplementary thing to understand about this:

The {\sl pool} file can be edited, for example to put some translation
for the records relating to information to user. But not all the records
can be tampered with since some strings are not dedicated to the user
but have a special definition because they are used in the processing
(example: ``13m2d5c2l5x2v5i'' used for converting numbers in roman
numerals).

But even when modifying the (possible) strings, the checksum shall not
be modified. And there has to be the same number of records in the same
order.

This file is indeed loaded by {\tt INITEX}, not by an instance of {\tt
VIRTEX}, since the latter loads a memory dump created by the former,
dump that have these preloaded strings.

The consequence, too, is that to have ``localized'' messages for \TeX\
and al., one would have to create---the way it is now---as many dumps as
there are languages.

@ \TeXprog is also a lesson on engineering. The word {\sl
control} comes from the french {\it contre r\^ole} that has given by
shrinking {\it contr\^ole} i.e. when there was a list of items (a {\it
r\^ole}), one could afterwards verify that what had been done or was
about to be done matched the reference {\it r\^ole}; verification was a
comparison between achieved or in progress {\sl against} ({\it contre})
the {\it r\^ole}).

Here, \TeX\ is an implementation of what is described in the \TeXbook.
The \TeXbook\ is the {\it r\^ole}. So one has to use and to refer to
both books to understand what is the aim (the \TeXbook) and what is the
way (\TeXprog).

You should have the two!

\enddiv %Pr�cis
