\def\mkfile#1#2{%
  \immediate\openout1{#1}%
  \toks0{#2}%
  \immediate\write1{\the\toks0 }%
  \immediate\closeout1 }

\mkfile{filea.tex}{\message{I'm file A}}
\mkfile{fileb.tex}{filea\noexpand}
\mkfile{y.tex}{\message{ignored}}

\scrollmode
\def\msg#1{\immediate\write0{>> #1 (\line):}}
\def\line{l. \the\inputlineno}

\msg{Basics}
\input{filea}

\msg{Missing file}
\input{filex}

% web2c prints double quotes around the file name in
% error messages if the name has spaces
\msg{Quotes in error messages}
\input{file x}

\def\x{filea}
% before the brace, stuff is expanded and \relax and spaces are ignored
% (@<Get the next non-blank non-relax non-call...@>;)
\msg{Expansion and fillers}
\input\relax\space\expandafter{\x}

% if another non-expandable token is found instead of the brace,
% the old filename scanner is used
\msg{Space-delimited}
\input\relax\expandafter\space\x

% The braced-input code is somewhat incompatible with Knuth's original
% version (though he classifies that as a system-dependent extension,
% so a change is acceptable).  An empty file name (.tex) can no longer
% be input with |\input\relax|.
\msg{Backwards incompatibility}
\input\relax filea

% With brace-delimited, you can nest \input (which makes
% arguably zero sense :)
\msg{Nesting}
\input{\input{fileb}}

% A space-delimited \input also works inside a brace-delimited one
\msg{Nesting again}
\input{\input fileb }

% An implicit \bgroup works, but not an implicit \egroup
\msg{Implicit bgroup}
\input \bgroup filea}

% Non-expandable tokens are passed through
\msg{Non-expandable tokens}
\input{file\relax a}

% Braces are treated as part of the file name as usual
\msg{Braces}
\input{f{i{l}e}a}

% Double are (unfortunately) dropped entirely when in a braced file
% name, and have no effect on the treatment of spaces
\msg{Quotes in braced file name}
\input{"f"i"l"e"a"}

% Line breaks are tokenised as spaces, contrary to space-delimited \input
\msg{Line breaks}
\input{file
a}

% Two line breaks are tokenised as \par
\msg{Blank line}
\input{file

a}

% And an \outer macro isn't accepted
\msg{\string\outer}
\input{file\bye
