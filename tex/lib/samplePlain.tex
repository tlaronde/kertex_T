% $Id: samplePlain.tex,v 1.2 2020/11/10 20:33:41 tlaronde Exp $
% This test file takes D. E. Knuth's example in the TeXbook, Appendix
% B, plus some examples taken from the TeXbook and some hints to other
% dvips(1) related extensions available in the core kerTeX.
%
% I have put my modifications, in the commentaries inside square
% brackets.
%
% The additions are tagged by commentary sentries.
%
% The definitive reference is the TeXbook.
% TL 2020
\headline={\hfil\TeX{}book examples and supplementary PS tricks}

% The original is not A4. We adapt for our additions to fit in one A4
% page but the elegance of the initial layout is partly lost: the page
% is too long for its width; but an augmented width would be ugly. This
% is a compromise.
%
\special{papersize=210mm,297mm} % A4 hint for the DVI driver
\hsize=30pc % The lines in this book are [30] picas wide.
\vsize=52pc % The page body is [this] (not counting footlines).


% D. E. Knuth's example starts below.

% This test file generates the output shown [when processed with a
% DVI driver (dvips here) and displayer].
% It's a bit complex because it tries to illustrate lots of stuff.
% TeX ignores commentary (like this) that follows a `%' sign.
 
% First the standard output style is changed slightly:
\footline={\tenrm Footline\quad\dotfill\quad Page \folio}
\pageno=1009 % This is the starting page number (don't ask why).
% See Chapter 23 for the way to make other page format changes via
% \hoffset, \voffset, \nopagenumbers, \headline, or \raggedbottom.
 
%\topglue 1in % This makes an inch of blank space (1in=2.54cm).
\centerline{\bf A Bold, Centered Title}
\smallskip % This puts a little extra space after the title line.
\rightline{\it avec un sous-titre \`a la fran\c caise}
% Now we use \beginsection to introduce part 1 of the document.
\beginsection 1. Plain \TeX nology % The next line must be blank!

The first paragraph of a new section is not indented.
\TeX\ recognizes the end of a paragraph when it comes to a blank
line in your manuscript file. % or to a `\par': See below.

Subsequent paragraphs {\it are\/} indented.\footnote*{The amount
    of indentation can be changed by changing a parameter called
{\tt\char`\\parindent}. [\dots]}
(See?) The computer breaks a paragraph's
text into lines in an interesting way---see reference~[1]---and h%
     yphenates words automatically     when necessary.
 
\midinsert % This begins inserted material, e.g., a figure.
\narrower\narrower % This brings the margins in (see Chapter 14).
\noindent \llap{``}If there hadn't been room for this material on
the present page, it would have been inserted on the next one.''
\endinsert % This ends the insertion and the effect of \narrower.
 
\proclaim Theorem T. The typesetting of $math$ is discussed in
Chapters 16--19, and math symbols are summarized in Appendix~F.
 
 % [And some example (19.5)].
 $$\prod_{k\ge0}{1\over(1-q^kz)}=
	\sum_{n\ge0}z^n\bigg/\!\!\prod_{1\le k\le n}(1-q^k).\eqno(16')$$

% ADDITION to D.E.K.'s example.
\beginsection 2. Tables\par

$$\hbox to\hsize{%
\vbox{\tabskip=0pt \offinterlineskip
\def\tablerule{\noalign{\hrule}}
\halign to135pt{\strut#&\vrule#\tabskip=1em plus2em&
  \hfil#&\vrule#&\hfil#\hfil&\vrule#&
  \hfil#&\vrule#\tabskip=0pt\cr\tablerule
&&\multispan5\hfil AT\&T Common Stock\hfil&\cr\tablerule
&&\omit\hidewidth Year\hidewidth&&
 \omit\hidewidth Price\hidewidth&&
 \omit\hidewidth Dividend\hidewidth&\cr\tablerule
&&1971&&41--54&&\$2.60&\cr\tablerule
&&   2&&41--54&&2.70&\cr\tablerule
&&   3&&46--55&&2.87&\cr\tablerule
&&   4&&40--53&&3.24&\cr\tablerule
&&   5&&45--52&&3.40&\cr\tablerule
&&   6&&51--59&&.95\rlap*&\cr\tablerule
\noalign{\smallskip}
&\multispan7* (first quarter only)\hfil\cr
}}\hfil
\vbox{\tabskip=0pt \offinterlineskip
\def\tablerule{\noalign{\hrule}}
\halign to200pt{\strut#&\vrule#\tabskip=1em plus2em&
  \hfil#&\vrule#&\hfil#\hfil&\vrule#&
  \hfil#&\vrule#\tabskip=0pt\cr\tablerule
&&\multispan5\hfil AT\&T Common Stock\hfil&\cr\tablerule
&&\omit\hidewidth Year\hidewidth&&
 \omit\hidewidth Price\hidewidth&&
 \omit\hidewidth Dividend\hidewidth&\cr\tablerule
&&1971&&41--54&&\$2.60&\cr\tablerule
&&   2&&41--54&&2.70&\cr\tablerule
&&   3&&46--55&&2.87&\cr\tablerule
&&   4&&40--53&&3.24&\cr\tablerule
&&   5&&45--52&&3.40&\cr\tablerule
&&   6&&51--59&&.95\rlap*&\cr\tablerule
\noalign{\smallskip}
&\multispan7* (first quarter only)\hfil\cr}}}$$

\beginsection 3. PS (and PDF) tricks

\input epsf
\input hyperbasics
% Scale eps image to 1 true inch horizontaling. Y will be scaled 
% accordingly.
%
\epsfxsize=6pc

% Put image, knowing its size, along with text.
%
\line{\hfil\epsffile[0 0 221 292]{quixante.eps}\hfil
	\vbox to 6pc{\hsize=23pc\parindent=10pt \parskip=0pt plus 1fill
This is an image put along with some text. This uses plain.tex macros,
\TeX\ primitives and Tomas Rokicki's {\tt epsf} macros and examples.

Everything used here is available in the core ker\TeX, including
hyperlinks. But this does not exercize, neither the full power and
versatility of \TeX, nor all that is available with ker\TeX\ core. Just
few examples\dots (See \href{http://kertex.kergis.com/}{the kertex Web
site for more}.)
	} % vbox
} %line for putting image and text along.
% NOITIDDA to D.E.K.'s example.

\beginsection 4. Bibliography\par % `\par' acts like a blank line.
\frenchspacing % (Chapter 12 recommends this for bibliographies.)
\item{[1]} D.~E. Knuth and M.~F. Plass, ``Breaking paragraphs
into lines,'' {\sl Softw. pract. exp. \bf11} (1981), 1119--1184.
\bye % This is the way the file ends, not with a \bang but a \bye.
